\documentclass[12pt,a4paper]{article}
\usepackage[legalpaper, portrait, lmargin=1cm, rmargin=1cm, tmargin=2cm, bmargin=2cm]{geometry}
\usepackage{fancyhdr}
\usepackage{amsmath}
\usepackage{amssymb}
\usepackage{graphicx}
\usepackage{wrapfig}
\usepackage{blindtext}
\usepackage{hyperref}
\usepackage{pdflscape}
\usepackage{svg}
\usepackage[most]{tcolorbox}
\usepackage{xcolor}
\usepackage[T1]{fontenc}


\graphicspath{ {./} }

\definecolor{linkcolor}{HTML}{1588e0}

\hypersetup{
  colorlinks=true,
  allcolors=linkcolor,
  pdftitle={Relatório AMS - Entrega 1 - 2024/2025},
  pdfpagemode=FullScreen,
}

\definecolor{bg}{rgb}{1,0.96,0.9}

\pagestyle{fancy}
\fancyhf{}
\rhead{Grupo \textbf{50}}
\lhead{Relatório Entrega 1 AMS 2024/2025 LEIC-A}
\cfoot{Francisca Almeida (105901), Guilherme Filipe (106326) e José Frazão (106943)}

\renewcommand{\footrulewidth}{0.2pt}

\renewcommand{\labelitemii}{$\circ$}
\renewcommand{\labelitemiii}{$\diamond$}


\begin{document}
\begin{titlepage}
  \begin{center}
    \vspace*{5cm}

    \Huge
    \textbf{Projeto AMS - Entrega 1}

    \vspace{0.5cm}
    \LARGE
    Grupo 50 | Turno L07 | LEIC-A

    \vspace{0.5cm}
    \large
    Prof. Luís Moreira de Sousa

    \vfill
  \end{center}
  \large
  \begin{itemize}
    \item[] \textbf{Francisca Almeida} (105901) - ?h
    \item[] \textbf{Guilherme Filipe} (106326) - ?h
    \item[] \textbf{José Frazão} (106943) - ?h
  \end{itemize}
\end{titlepage}

{\fontfamily{qhv}\selectfont
\begin{center}
\begin{large}
    \textbf{Texto UoD da proposta de Oportunidade}
\end{large}
\end{center}

\begin{small}
\begin{normalsize}
\textbf{Oportunidade}
\end{normalsize}\\
O grande aumento do número de empresas que procuram melhorar os seus ecossistemas através da integração de serviços em cenário de transformação digital criou uma excelente oportunidade para a \textbf{BGB} expandir a sua oferta.\\
Com isso em mente e aproveitando a aposta da \textbf{UE} nessa mesma transformação digital, através da disponibilização da \textbf{EUDIW}, a \textbf{BGB} decidiu lançar o seu novo produto, o \textbf{BioBoxPlus}.\\
O \textbf{BioBoxPlus} não é um produto totalmente novo mas sim o produto já existente \textbf{BioBox} acrescido de uma nova aplicação, a \textbf{BEUDIW}, e da capacidade de ler as \textbf{EUDIW}.\\
Após contratar uma equipa de consultores externos, este sugeriram à \textbf{BGB} que aproveitasse esta oportunidade e integrasse este novo produto no mercado de transporte de passageiros.\\

\begin{normalsize}
\textbf{A aplicação BEUDIW}
\end{normalsize}\\
A aplicação \textbf{BEUDIW} é a componente que realmente acrescenta valor a este novo produto.\\
A \textbf{BEUDIW} é altamente customizável, sendo desenvolvida para cada cliente consoante as suas necessidades e desejos.\\
Este processo de customização da app ocorre numa fase inicial durante as negociações do contrato e, posteriormente, existe um diálogo permanente entre o cliente e a BGB de forma a garantir a implementação de novas funcionalidades sempre que possível, sendo estas analisadas pelas próprias equipas existentes na \textbf{BGB}.\\
A aplicação é instalada num novo servidor, diferente do servidor da \textbf{BGM}, de forma a maximizar a segurança e integridade dos dados dos utilizadores.\\
Como funcionalidade base, a \textbf{BEUDIW} permite a criação de relatórios totalmente personalizáveis sobre os acessos aos \textbf{BGate}, tal como já era possível com o \textbf{BioBox}.\\
No entanto, com a possibilidade de leitura da \textbf{EUDIW}, os relatórios podem também conter informações adicionais que possam ser extraídas da carteira digital do utilizador.\\
Além disso, a \textbf{BEUDIW} também permite à partida a criação de credenciais personalizadas que podem ser utilizadas, por exemplo, para garantir o acesso de funcionários da empresa a certas áreas restritas definidas pela mesma.\\

\begin{normalsize}
\textbf{Integração da EUDIW}
\end{normalsize}\\
Com o intuito de alavancar os processos de transformação digital dos seus clientes, o \textbf{BioBoxPlus} permite agora a leitura da \textbf{EUDIW} para validação de acessos.\\
Este novo tipo de validação possibilita uma autenticação centralizada e automática de cada utilizador.\\
Adicionalmente, o uso da \textbf{EUDIW} não se restringe apenas à verificação se o utilizador tem o bilhete necessário para aquele acesso, permitindo também a aplicação automática de descontos, por ex. através da confirmação da idade do utilizador, ou o pagamento no local, entre outros.\\
No entanto, a disponibilização do uso da \textbf{EUDIW} acarreta uma necessidade de cuidado especial no que toca à manipulação dos dados dos utilizadores.\\
De forma a garantir conformidade com o \textbf{RGPD}, desenhado pela \textbf{EU} para regular a proteção dos dados pessoais, a \textbf{BEUDIW} só acede a dados do utilizador com o seu explícito consentimento.\\
Além disso, todos estes dados são anonimizados antes de serem guardados no servidor e não excedem o tempo máximo permitido por lei para seu armazenamento.\\
A \textbf{BGB}, com o intuito de garantir uma conformidade total com a lei no que toca à manipulação deste tipo de dados, contratou uma equipa externa de consultores que regularmente analisa as práticas da empresa e sinaliza algum comportamento ilegal.\\
\end{small}
}

\begin{landscape}
  \begin{figure}
    \centering
    \includesvg[inkscapelatex=false,width=1.2\textwidth]{assets/A1-1.svg}
    \caption{Diagrama de Contexto do Produto BioBox}
    \label{fig:A1-1}
  \end{figure}
\end{landscape}

\begin{landscape}
  \begin{figure}
    \centering
    \includesvg[inkscapelatex=false,width=1.4\textwidth]{assets/A2-1.svg}
    \caption{Diagrama da Vista Geral do Produto BioBox}
    \label{fig:A2-1}
  \end{figure}
\end{landscape}

\begin{landscape}
  \begin{figure}
    \centering
    \includesvg[inkscapelatex=false,width=1.35\textwidth]{assets/A3-1.svg}
    \caption{Diagrama de Contexto do Produto BioBoxPlus}
    \label{fig:A3-1}
  \end{figure}
\end{landscape}

\begin{landscape}
  \begin{figure}
    \centering
    \includesvg[inkscapelatex=false,width=1.35\textwidth]{assets/A4-1.svg}
    \caption{Diagrama da Vista Geral do Produto BioBoxPlus}
    \label{fig:A4-1}
  \end{figure}
\end{landscape}

\begin{landscape}
  \begin{figure}
    \centering
    \includesvg[inkscapelatex=false,width=1.35\textwidth]{assets/B1-1.svg}
    \caption{Diagrama Privado do Processo de Execução de Pedido}
    \label{fig:B1-1}
  \end{figure}
\end{landscape}

\begin{landscape}
  \begin{figure}
    \centering
    \includesvg[inkscapelatex=false,width=0.89\textwidth]{assets/B2-1.svg}
    \caption{Diagrama de Colaboração do Processo de Execução de Pedido}
    \label{fig:B2-1}
  \end{figure}
\end{landscape}

\end{document}